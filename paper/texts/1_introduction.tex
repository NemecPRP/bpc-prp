\section{Úvod}





Navigace autonomních robotů v neznámém prostředí je jedním z klíčových problémů řešených v oblasti robotiky. Efektivní orientace v prostoru bez předchozí znalosti prostředí je zásadní u průzkumných nebo záchranných misí.

Cílem tohoto projektu bylo navrhnout a implementovat algoritmus řízení univerzitou poskytnutého robota tak, aby byl schopen samostatně projít celým bludištěm za co nejrychlejší čas bez jakéhokoliv dotyku nebo kolizí se stěnou bludiště.

Bludiště bylo sestaveno jako nepravidelná síť chodeb sestavená z buněk o rozměrech 40x40 cm a ven vedla pouze jedna trasa. Nacházely se zde i slepé uličky, penalizace ve formě "setkání s Minotaurem" a časový bonus při nalezení pokladu. Na zemi byly umístěny aruco tagy, které určovaly nejkratší únikovou trasu z bludiště na křižovatkách.

Cílem této práce je popsat návrh implementaci a výsledky námi navrženého algoritmu řízení robota v diskrétním prostoru, který mu umožnil utéct z bludiště.
